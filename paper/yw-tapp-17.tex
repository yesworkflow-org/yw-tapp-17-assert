%\documentclass[letterpaper,twocolumn,10pt]{article}
\documentclass[nocopyrightspace]{sigplanconf}

%\usepackage{usenix}
\usepackage{amssymb}
\usepackage{amsmath}
\usepackage{amsthm}
\usepackage{amsfonts}
\usepackage{multirow}
\usepackage{epsfig}
\usepackage{color}
\usepackage{soul}

\usepackage{float}
\usepackage{subfigure}
\usepackage{color} 
\usepackage{graphicx}
\usepackage{xspace}
\usepackage{url}
\usepackage[colorlinks=true,
           urlcolor=blue,
           citecolor=blue,
           linkcolor=blue,
           % bookmarksopen,
           % bookmarksopenlevel=1,
           % pdftitle={},
           % pdfauthor={Joe Doe},%
           % pdfsubject={},%
           % pdfkeywords={provenance, workflows},
           bookmarks=false,
           bookmarksnumbered,% For PDF navigation
           % backref=page,
           % pagebackref=page,
           linktocpage=true
           ]{hyperref}

\usepackage[utf8]{inputenc}

% Default fixed font does not support bold face
\DeclareFixedFont{\ttb}{T1}{txtt}{bx}{n}{9} % for bold
\DeclareFixedFont{\ttm}{T1}{txtt}{m}{n}{9}  % for normal

% Custom colors
%\usepackage{color}
\definecolor{deepblue}{rgb}{0,0,0.5}
\definecolor{deepred}{rgb}{0.6,0,0}
\definecolor{deepgreen}{rgb}{0.3,0.5,0.3}

\definecolor{ignore}{rgb}{.5,.5,.5}
\definecolor{while}{rgb}{0.95,0.95,0.95}

\usepackage{listings}

% Python style for highlighting
\newcommand\pythonstyle{\lstset{
    language=Python, 
    columns=fullflexible,
    basicstyle=\color{ignore}\ttm,%\small\sffamily, %\ttm,  
    commentstyle=\color{deepblue}\ttb,
    emph={raw_image_path},          % Custom highlighting
    emphstyle=\ttb\color{deepblue},    % Custom highlighting style
    otherkeywords={self},             % Add keywords here
    % keywordstyle=\ttb\color{ignore},
    % stringstyle=\color{deepgreen},
   frame=lines,
    showstringspaces=false,
    numbers=left,   
    firstnumber=1,
    numberfirstline=true,
%    numbersep=5pt
}}

% Python environment
\lstnewenvironment{python}[1][]
{
\pythonstyle
\lstset{#1}
}
{}

% Python for external files
\newcommand\pythonexternal[2][]{{
\pythonstyle
\lstinputlisting[#1]{#2}}}

% Python for inline
\newcommand\pythoninline[1]{{\pythonstyle\lstinline!#1!}}

\usepackage{algorithm}
\usepackage{verbatim} 

\newcommand{\TODO}[1]{\textcolor{red}{\textbf{TODO:}}\textcolor{blue}{\xspace#1}}
\newcommand{\Figref}[1]{Figure\,\ref{#1}}
\newcommand{\figref}[1]{Fig.\,\ref{#1}}

\newcommand{\data}[1]{\ensuremath{\mathtt{#1}}\xspace}
\newcommand{\V}{\ensuremath{V}\xspace}
\newcommand{\E}{\ensuremath{E}\xspace}
\newcommand{\D}{\ensuremath{\mathrm{D}}\xspace}   % Data items as subset of \V
\newcommand{\I}{\ensuremath{\mathrm{I}}\xspace}   % Data items as subset of \V

%\renewcommand*{\thefootnote}{\fnsymbol{footnote}}

%\newcommand{\code}[1]{\ensuremath{\mathsf{#1}}}
\newcommand{\code}[1]{\ensuremath{\mathtt{#1}}}
\newcommand{\term}[1]{\ensuremath{\mathsf{#1}}}


\newcommand{\NW}{\textsf{noWorkflow}}
\newcommand{\YW}{\textsf{YesWorkflow}}
\newcommand{\yw}{\textsf{YW}}

\newcommand{\YWT}{\textsf{YesWorkflow}}
\newcommand{\ywt}{\textsf{YW}}

\newcommand{\ywa}[1]{\texttt{#1}}
\newcommand{\ywm}[1]{\texttt{#1}}

\newcommand{\R}{\textsf{R}}
\newcommand{\MATLAB}{\textsf{MATLAB}}

\begin{document}

\title{Validating YesWorkflow annotations using declarations of 
       fine-grained dependencies of script outputs on script inputs}

\authorinfo{
  \begin{minipage}{1.0\linewidth}
    \begin{minipage}{0.24\linewidth} \centering
      Timothy McPhillips 
    \end{minipage}
    \begin{minipage}{0.24\linewidth} \centering
      Qian Zhang 
    \end{minipage}
    \begin{minipage}{0.24\linewidth} \centering
      Bertram Lud\"{a}scher
    \end{minipage}
  \end{minipage}
}
{
  \begin{minipage}{1.0\linewidth}
    \begin{minipage}{0.24\linewidth} \centering
      University of Illinois (UIUC)
    \end{minipage}
    \begin{minipage}{0.24\linewidth} \centering
      University of Illinois (UIUC)
    \end{minipage}
    \begin{minipage}{0.24\linewidth} \centering
      University of Illinois (UIUC)
    \end{minipage}    
  \end{minipage}
}
{
$\,$\\
  \vspace{-4mm}
  \begin{minipage}{1.0\linewidth}
    \begin{minipage}{0.24\linewidth} \centering
      tmcphillips@absoluteflow.org
    \end{minipage}
    \begin{minipage}{0.24\linewidth} \centering
      zhangqian06@gmail.com
    \end{minipage}
    \begin{minipage}{0.24\linewidth} \centering
      ludaesch@illinois.edu
    \end{minipage}
  \end{minipage}
}

\maketitle

\begin{abstract}

  YesWorkflow is an annotation language for declaring the dataflow 
  structures otherwise hidden in scripts implementing scientific 
  workflows. An accompanying software toolkit extracts these annotations, 
  constructs workflow models corresponding to annotated scripts, and 
  renders the resulting models graphically. YesWorkflow visualizations 
  reveal the computation steps involved in producing each script output 
  nd the paths taken by data through those steps during a script run. 
  By exposing these models as Datalog facts, YesWorkflow enables logic 
  programs both to query the workflow graph and to produce additional 
  visualizations that highlight how specific outputs are computed by 
  \a script. These capabilities require not just that the individual 
  YesWorkflow annotations within a script be correct, but also that the 
  collection of annotations in a script be both internally consistent 
  and sufficiently complete to support these queries. Here we describe 
  a way for researchers to confirm the completeness and internal 
  consistency of the YesWorkflow models implied by the annotations in 
  their scripts. The approach employs new YesWorkflow annotations that 
  the researcher applies to the script as a whole to declare the 
  fine-grained dependencies of each script output on specific script 
  inputs. Because annotations on the individual steps comprising a script 
  are applied independently of these new, script-level data-dependency 
  declarations, the latter can be used to validate the former.

\end{abstract}

\section{Introduction} \label{intro}

\section{Conclusions} 

\paragraph{Acknowledgments.}
Work supported in part by the National Science Foundation under awards
DBI-1356751 (Kurator), SMA-1439603 (SKOPE), ACI-0830944 (DataONE).

\bibliographystyle{alpha-initials-big}
\bibliography{yw-tapp-17}

\end{document}







